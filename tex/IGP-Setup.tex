%!TEX root = ../BGP_for_networks_who_peer.tex
\chapter{Setting up an \gls{IGP}}
\section{Motivation}
\gls{BGP} never runs alone. To distribute the IP addresses of interfaces
within an \gls{AS}, an \glsreset{IGP}\gls{IGP} is needed.

There are several options which \gls{IGP} to use, but normally you will not be free
to choose. Either someone has already made the decision, or you are restricted
on what your router vendor supports.

If you really have several options, keep in mind the following points when
making your decision:
\begin{description}
  \item[Openness:] The IGP must be supported not only by the routers you are using currently, but also by the routers you are purchasing over the next few years.
  \item[IPv6 awareness:] Even if you are currently not using IPv6, you will some time in the future. Your IGP should either support IPv6 or you should be able to run a 2nd IGP in parallel for IPv6.
\end{description}

\section{Configuring OSPF}
\gls{OSPF} runs on top of IP. So you need to configure IP addresses for your
interfaces, and then define the use of OSPF on these IPs. It really helps
if you have all your interface IP addresses out of the same network block
(and use this block for nothing else).

Also, OSPF has the concept of \emph{areas}: You have to define at least one area, called either backbone or area 0 (zero).

Example (Cisco):\label{example-ospf}
\begin{verbatim}
interface GigabitEthernet1/0
 ip address 192.168.2.2 255.255.255.252
!
router ospf 64500
 network 192.168.2.0 0.0.0.255 area 0
\end{verbatim}

This would switch on OSPF on all interfaces with an IP address within 192.168.2.0/24.

Same example for Mikrotik routers:
\begin{verbatim}
/ip address
add address=192.168.2.2/30 interface=ether1 network=192.168.2.0

/routing ospf network
add area=backbone network=192.168.2.0/24
\end{verbatim}

Same example for FRRouting software router:
\begin{verbatim}
interface wlan0.101
  ip address 192.168.2.2/30
!
router ospf
  network 192.168.2.2/30 area 0
\end{verbatim}

On Juniper you enable OSPF on the interfaces:
\begin{verbatim}
  interfaces {
      em0 {
          unit 0 {
              family inet {
                  address 192.168.2.2/30;
              }
          }
      }
  }
  protocols {
    ospf {
      area 0.0.0.0 {
        interface em0.0;
      }
    }
  }
\end{verbatim}

Keep in mind that OSPFv2 (the ``standard'' OSPF version) supports IPv4 only. This is completely fine; for IPv6 you can use OSPFv3 to keep things separate. Or you can run IS-IS which supports IPv4 and IPv6.

\section{Configuring IS-IS}
\gls{IS-IS} does not run on top of IP but runs directly on layer 2 (Ethernet or other). So you do not have to define the network it runs on, but the interfaces you want it enabled on.

\emph{Important}: The protocols you enable it for must be the same on both sides of a connection. So if you enable IS-IS for IPv4 and IPv6 on one interface, you must do so as well on the connected interface.

Example for IS-IS on Cisco:
\begin{verbatim}
interface GigabitEthernet0/0
  ip address 192.168.2.2 255.255.255.252
  ipv6 address autoconfig
  ip router isis
  ipv6 router isis
!
router isis
  net 49.0000.0000.0000.0001.00
\end{verbatim}
The number behind the ``net'' statement is the router-id and must be unique within your network.

Mikrotik does not support IS-IS.

Example for IS-IS on FRRouting:
\begin{verbatim}
router isis myname
  net 49.0000.0000.0000.0001.00
!
interface eth1
  ip address 192.168.2.2/30
  ipv6 address autoconfig
  ip router isis myname
  ipv6 router isis myname
\end{verbatim}

Example for IS-IS on Juniper is a little bit more complex. For the loopback interface \emph{lo0} we configure a static IPv6 address, for the ethernet interface we use auto-configuration. Also, the router-id goes into the loopback interface and \emph{family iso} has to be enabled on all interfaces where IS-IS is to be used.
\begin{verbatim}
  interfaces {
      em0 {
          unit 0 {
              family inet {
                  address 192.168.2.2/30;
              }
              family iso;
              family inet6;
          }
      }
      lo0 {
          unit 0 {
              family inet {
                  address 192.168.1.3/32;
              }
              family inet6 {
                address 2001:db8:500::1:3/128;
              }
              family iso {
                  address 49.0000.0000.0000.0003.00;
              }
          }
      }
  }
  protocols {
    isis {
        interface em0.0;
        interface lo0.0;
    }
\end{verbatim}


\section{Configuring OSPFv3}
To integrate IPv6 a new version of OSPF was necessary - this came into being as OSPFv3. Except for virtual links, OSPFv3 uses IPv6 link-local addressing. It is configured on the interface, not on a network like OSPFv2.

The only remains of 32bits are area and router ids - they are still 32bit like IPv4 addresses.

Example for OSPFv3 on Cisco IOS:
\begin{verbatim}
interface Loopback0
  ipv6 address 2001:DB8:500::1:3/128
  ipv6 ospf 64500 area 0
!
interface GigabitEthernet0/0
  ipv6 address autoconfig
  ipv6 ospf 64500 area 0
\end{verbatim}
In this example, \emph{64500} is the process id of the OSPFv3 process - you can choose any number. It is not even necessary to define the process itself - simply enable OSPv3 on all interfaces where you want to use it. Whether you want to use auto-configured IPv6 addresses on your link interfaces or static IPv6 addresses is up to you.

Example for Mikrotik:
\begin{verbatim}
/ipv6 address
  add address=2001:DB8:500::1:1/128 interface=loopback0
  add interface=ether0 address=2001:DB8:496::1:2/126

/routing ospf-v3 interface
  add area=backbone interface=loopback0
  add area=backbone interface=ether0
\end{verbatim}

Example for FRRouting:
\begin{verbatim}
router ospf6
  router-id 192.168.1.1
  interface dummy0 area 0.0.0.0
  interface eth1 area 0.0.0.0
  interface eth2 area 0.0.0.0
\end{verbatim}
You have to configure an (IPv4-like) router-id manually - the best approach is to use the IPv4 address of your Loopback interface. Also the area notation is written like an IPv4 address.

Example for Juniper:
\begin{verbatim}
  protocols {
    ospf3 {
        area 0.0.0.0 {
            interface lo0.0;
            interface em0.0;
        }
    }
  }
\end{verbatim}

\iffalse % ---------------- do not print ----------------------

\section{Experiment 1a, 1b and 1c: Setup IGP}
Please see separate experiment documentation.

\section{Experiment \theexperiment}
\subsection{Network Setup}
Network is a ring-shaped network out of all participants' routers (see
picture \ref{fig:exp1}).

\begin{figure}
  \includegraphics[width=\linewidth]{img/i-wt-BGP-Raspi-Experiment01-iBGP-01.png}
  \caption{Experiment 1 - Setting up an IGP}
  \label{fig:exp1}
\end{figure}

\subsection{Task}
To complete this experiment, you must:
\begin{itemize}
  \item Configure your assigned loopback IP address on your router
  \item Configure \gls{OSPF} on your router so it talks to its neighbors
  \item Redistribute your loopback address via OSPF. Use
  \begin{verbatim}
    redistribute connected subnets
  \end{verbatim}
  for this.
  \item See the other routers loopback addresses in your routing table.
\end{itemize}

\fi % ------------------------------------
